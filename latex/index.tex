{\itshape R\+E\+A\+D\+ME is located at \href{https://gitlab.epfl.ch/majoor/project-5-numerical-integration}{\tt https\+://gitlab.\+epfl.\+ch/majoor/project-\/5-\/numerical-\/integration}. If images fail to load, view the R\+E\+A\+D\+ME online.}

\subsubsection*{Compiling the Program}

The following steps should be undertaken to run our code
\begin{DoxyItemize}
\item clone the repository with 
\begin{DoxyCode}
git clone git@gitlab.epfl.ch:majoor/project-5-numerical-integration.git
\end{DoxyCode}

\item move to the cloned directory 
\begin{DoxyCode}
cd project-5-numerical-integration/
\end{DoxyCode}

\item fetch the submodules (Eigen and Googletest) using 
\begin{DoxyCode}
git submodule update --init
\end{DoxyCode}

\item build the executable using C\+M\+A\+KE 
\begin{DoxyCode}
mkdir build
cd build
cmake ..
\end{DoxyCode}

\item run the executable 
\begin{DoxyCode}
./integration
\end{DoxyCode}

\end{DoxyItemize}

This code will produce the following output\+: 
\begin{DoxyCode}
Successfully opened file ../readfile.txt

---Input Data---

D: 2
m: 2
boundsX: 
1 2
6 7
boundsY: 
  5   6
  8 9.5
noSteps: 
10 11
 3 10
coefficients: 
  (1,1) (15,-5) (-2,11) (1,3.6)
 (5,-6)  (10,1)   (0,0)   (0,0)

---Integration Methods---

MidpointFormula:
[ 199.819 + 388.072i,
125 + -3.75i ]

TrapezoidalRule:
[ 202.09 + 390.886i,
125.909 + -3.65909i ]

SimpsonsRule:
[ 199.833 + 388.125i,
125 + -3.75i ]


Process finished with exit code 0
\end{DoxyCode}


\subsubsection*{Configuring the Program}

To run the program with different integrals and different domains of integration, the input file {\ttfamily readfile.\+txt} can be configured. An example of the format required is shown below\+:



The first line details the length of inputs the program should expect. The lines (a), (b), (c) and (d) are then read as follows\+:
\begin{DoxyItemize}
\item (a) first domain is between x=1 and x=2, y=5 and y=6 and the integration method undergoes 10 steps in the x direction and 11 steps in the y direction
\item (b) first domain is between x=6 and x=7, y=8 and y=9.\+5 and the integration method undergoes 3 steps in the x direction and 10 steps in the y direction
\item (c) first function output reads 1+1i + (15-\/5i)x + (-\/2+11i)y + (1+3.6i)x$^\wedge$2
\item (d) second function output reads 5-\/6i + (19+1i)x + (0+0i)y + (0+0i)x$^\wedge$2
\end{DoxyItemize}

\subsubsection*{Typical Program Usage}

A typical execution would be to edit the configuration file {\ttfamily readfile.\+txt} to indicate a new integration problem, and then to simply rerun the program with {\ttfamily ./integration} and view the results of the methods.

\subsubsection*{Program Features}


\begin{DoxyItemize}
\item Reads arbitrary input from file
\item Implementation of 3 integration methods in 2D and 1D
\item Error handling
\item Testing of all features
\item Documentation of all code
\item Extensible code due to polymorphic approach
\end{DoxyItemize}

\subsubsection*{Documentation}

Documentation can be viewed in the {\ttfamily html/} folder. To open the documentation, once can execute the following\+: 
\begin{DoxyCode}
xdg-open html/index.html
\end{DoxyCode}


\subsubsection*{Tests}

Two types of test are run using googletest \+:
\begin{DoxyItemize}
\item Tests of the text file reader
\item Tests of the integration methods
\end{DoxyItemize}

For the reader, we execute a few death tests to ensure that the first inputs are asserted to be positive.

For the integration, we run tests on polynomials of 4 values of l (l is defined above).
\begin{DoxyItemize}
\item For l=1, we integrate a constant function and expect an exact solution for all 3 methods.
\item For l=3, we expect an exact solution for midpoint and Simpson but an approximate solution for trapezoidal.
\item For l=6, we expect an exact solution for Simpson but an approximate solution for midpoint and trapezoidal.
\item For l=9, we expect an approximate solution for all 3 methods.
\end{DoxyItemize}

\subsubsection*{Current Issues/\+Limitations}


\begin{DoxyItemize}
\item Limited error handling
\item Wasteful to have {\ttfamily Eigen\+::\+Vector\+Xcd ($\ast$f)(double x, double y, Eigen\+::\+Matrix\+Xcd \&coeff)} take {\ttfamily \&coeff} as input
\item No specifiers const and unsigned where appropriate
\item No use of override
\end{DoxyItemize}

\subsubsection*{Suggestions for Future Improvement}


\begin{DoxyItemize}
\item Extension to 3D or higher
\item Read input from other file structures
\item Store result in a file 
\end{DoxyItemize}