\subsubsection*{Compiling the Program}

The following steps should be undertaken to run our code
\begin{DoxyItemize}
\item clone the repository with {\ttfamily git clone git@gitlab.\+epfl.\+ch\+:majoor/project-\/5-\/numerical-\/integration.\+git}
\item fetch the submodules (Eigen and Googletest) using {\ttfamily git submodule update -\/-\/init}
\item build the executable using C\+M\+A\+KE 
\begin{DoxyCode}
mkdir build
cd build
cmake ..
\end{DoxyCode}

\item run the executable {\ttfamily ./integration}
\end{DoxyItemize}

This code will produce the following output\+: 
\begin{DoxyCode}
[to be copy pasted]
\end{DoxyCode}


\subsubsection*{Configuring the Program}

To run the program with different integrals and different domains of integration, the input file can be configured. An example of the format required is shown below\+:



The first line details the length of inputs the program should expect. The lines (a), (b), (c) and (d) are then read as follows\+:
\begin{DoxyItemize}
\item (a) first domain is between x=1 and x=2, y=5 and y=6 and the integration method undergoes 10 steps in the x direction and 11 steps in the y direction
\item (b) first domain is between x=6 and x=7, y=8 and y=9.\+5 and the integration method undergoes 3 steps in the x direction and 10 steps in the y direction
\item (c) first function output reads 1+1i + (15-\/5i)x + (-\/2+11i)y + (1+3.6i)x$^\wedge$2
\item (d) second function output reads 5-\/6i + (19+1i)x + (0+0i)y + (0+0i)x$^\wedge$2
\end{DoxyItemize}

\subsubsection*{Typical Program Usage}

\mbox{[}to be done\mbox{]}

\subsubsection*{Program Features}

\mbox{[}to be done\mbox{]}

\subsubsection*{Tests}

\mbox{[}to be done\mbox{]} 